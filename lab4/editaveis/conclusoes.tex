\section{Discuss�o e Conclus�es}
Os valores de tens�o medidos na carga no laborat�rio e os valores calculados te�ricamente foram muito pr�ximos, visto que a diferen�a entre eles foi menor que $10\%$. Esse valor � visto como muito baixo visto que os resistores usados para montagem do circuito possuem toler�ncia de $\pm 5\%$. Um fator que sugere esse erro foi que a imped�ncia de Theven�n teve um valor medido alto em rela��o ao calculado, em teoria o valor para resist�ncia da carga deveria ser de $87.5\Omega$ (171.61mW) para a m�xima tranfer�ncia de pot�ncia, na pr�tica essa valor aumentou para $94\Omega$ (143.68mW). Baseado nesses valores de erro muito baixos � poss�vel dizer que o experimento foi um sucesso.
